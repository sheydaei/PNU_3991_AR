\documentclass{article}
\usepackage{multicol}
\usepackage{graphicx}
\linespread{1.35}
\usepackage{amsmath}
\usepackage{color}
\usepackage{xcolor}
\usepackage{tikz-cd}
\usepackage{tikz}
\usetikzlibrary{arrows,automata}


\begin{document}

\begin{center}
\section{picture}
\includegraphics[width=13cm,height=3cm]{133-1.png}
\end{center}

\large{
\textbf{Introduction}\\
}

Chapter 3 already discusses the Mealy and Moore machines. These machines fall in the category of
automata with output. Finite state machine, in short FSM, is a machine with a finite number of states. As
all finite automata contain finite number of states, they are FSMs also. In this chapter, we shall discuss
elaborately the different features of an FSM, some application such as sequence detector, incompletely
specified machine, where some states and/or some outputs are not mentioned, definite machine, finite
machine, information lossless machine, minimization of machines, etc.\\

\vspace*{0.5cm}
\large{
\textbf{4.1 Sequence Detector}
}

The sequence detector detects and counts a particular sequence from a long input. Let us assume that,
in a circuit, a long input of ‘0’ and ‘1’ is fed. Let us also assume that, from there, it needs to count the
occurring of a particular sequence where overlapping sequences are also counted. The circuit generates
‘1’ if the particular output pattern is generated; otherwise, ‘0’ is generated.\\
\hspace*{0.5cm} The sequence detector can be designed either in a Mealy or Moore machine format.\\
\hspace*{0.5cm} The following examples describe the design of a sequence detector elaborately.\\

\vspace*{0.5cm}

\fcolorbox{red}{blue}{\textbf{\textcolor[rgb]{1.00,1.00,1.00}{Example 4.1}}}\hspace*{0.1cm} \texttt{Design a two-input two-output sequence detector which generates an output ‘1’ every
time the sequence 1001 is detected. And for all other cases, output ‘0’ is generated.
Overlapping sequences are also counted.}\\

\vspace*{0.5cm}
\textbf{Solution:} Before designing this circuit, some clarification regarding overlapping sequence is needed.
Let the input string be $1001001$.\\
\hspace*{0.5cm} We have to design the circuit in such a way that it will take one input at a time. The input can be either
'0' or '1' (two types of input). The output will also be of two types, either '0' or '1'. The circuit can store
(memorize) the input string up to four clock pulses from $t_i-3$ to $t_i$.\\
\hspace*{0.5cm} If the input string is placed according to clock pulses, the output will become\\

\begin{center}
\section{picture}
\includegraphics[width=9cm,height=1.7cm]{133-2.png}
\end{center}

\newpage
\begin{flushleft}
    \textbf{134}\hspace*{0.1cm} \textbf{$|$} \hspace*{0.1cm} \texttt{Introduction to Automata Theory, Formal Languages and Computation}
  \end{flushleft}

  \vspace*{0.5cm}
\hspace*{0.5cm} The first input at $t_1$ is 1, and as there is no input from $t_i-3$ to ti, the input sequence does not equal 1001.
So, the output will be 0. The same cases occur for the inputs up to $t_3$.\\
\hspace*{0.5cm} But at time $t_4$, the input from $t_i-3$ to $t_i$ becomes 1001, and so the output '1' is produced at time $t_4$. At
time $t_5$ and $t_6$, the input string from $t_i-3$ to $t_i$ are 0010 and 0100, respectively. So, the output '0' is produced.
But at $t_7$ clock pulse, the input string is 1001, and so the output '1' is produced. As the '1' at $t_4$ is
overlapped from $t_1$ to $t_4$ and from $t_4$ to $t_7$, this is called overlapping condition as shown in Fig. 4.1.\\

\hspace*{0.5cm} For this case, we have to fi rst draw the state diagram given in Fig. 4.2.\\

\begin{center}
\section{picture}
\includegraphics[width=9cm,height=4cm]{134.png}
\end{center}

\hspace*{0.5cm} In state $S_1$, the input may be either '0' or '1'. If the input is given '0', there is no chance to get 1001.
The machine has to wait for the next input. So, '0' becomes a loop on S1 with output '0'.\\
\hspace*{0.5cm} If the input is '1', there is a chance to get 1001 by considering that input, and so by getting input '1'
the machine moves to $S_2$ producing output '0'.\\
\hspace*{0.5cm} In $S_2$, again the input may be either '0' or '1'. If it is '1', the input becomes 11. There is no chance
to get 1001 by taking the previous inputs from $t_i-1$ to ti. But there is a chance to get 1001 by considering
the given input '1' at ti. So, it will be in state $S_2$. (If it goes to S1, then there will be a loss of clock pulse,
which means again from $S_1$ by taking input '1', it has to come to $S_2$, i.e., one extra input, which means
one clock pulse is needed, and for this the output will not be in right pattern.) If the input is '0', the
input becomes 10, by considering the previous input. As there is a chance to get 1001, it will move to $S_3$.\\

to get 1001, it shifts to $S_4$. But if it gets '0', it has no chance to get 1001 considering the previous input,
but there is a chance to get 1001 by considering the given input '1'. So, it will shift to $S_2$ as we know by
getting '1' in $S_1$ the machine comes to $S_2$.\\
\hspace*{0.5cm} In $S_4$, if it gets '0', the input will become 1000, but it does not match with 1001. So it has to start from the
beginning, i.e., S1. Getting '1', the string becomes the desired 1001. The overlapping part is the last '1' as
given in Fig. 4.1. From the last '1' of 1001, if it gets 001 only, it will give an output '1'. So, it will go to $S_2$.\\

\vspace*{0.2cm}
\textbf{State Table:} From the previous state diagram, a state graph can be easily constructed.\\
\begin{center}
\begin{tabular}{ccc}
 \hline

 \hline

 \hline

 \hline
 & \multicolumn{2}{c}{$Next State, O/P$}\\
 \cline{2-3}
 $State$ &  $X=0$ & $=1$\\
\hline
$S_1$ & $S_1, 0$  & $S_2, 0$\\
$S_2$ & $S_3, 0$  & $S_2, 0$\\
$S_3$ & $S_4, 0$  & $S_2, 0$\\
$S_4$ & $S_1, 0$  & $S_2, 1$\\
 \hline

 \hline

 \hline

 \hline
\end{tabular}
\end{center}

\newpage
\begin{flushright}
 \texttt{Finite State Machine} \hspace*{0.1cm}\textbf{$|$} \hspace*{0.1cm} \textbf{135}\hspace*{0.1cm}
\end{flushright}
\vspace*{0.5cm}

\textbf{State Assignment:} For making a digital circuit, the states must be assigned to some binary numbers.
This is called state assignment. As the number of states is 4, only two-digit is sufficient to represent the
four states $(2^2 = 4)$.\\
\hspace*{0.5cm} Let $S_1$ be assigned to 00, $S_2$ be assigned to $0_1$, $S_3$ be assigned to 11, $S_4$ be assigned to 10.\\
\hspace*{0.5cm} After doing this state assignment, the state table becomes\\

\begin{center}
\begin{tabular}{ccccc}
 \hline

 \hline

 \hline

 \hline
 &  \multicolumn{2}{c}{Next State, ($Y_1Y_2$)} &  \multicolumn{2}{c}{$O/P (z)$}  \\
  \cline{2-3}                         \cline{4-5}
 $Present State$($y_1y_2$) &  $X = 0$ & $= 1$ & $= 0$ &  $= 1$\\
\hline
$00$ & $00$ & $01$ & $0$ & $0$\\
$01$ & $11$ & $01$ & $0$ & $0$\\
$11$ & $10$ & $01$ & $0$ & $0$\\
$10$ & $00$ & $01$ & $0$ & $1$\\
 \hline

 \hline

 \hline

 \hline
\end{tabular}
\end{center}

From this state assignment table, the digital function can easily be derived.\\

\begin{center}
\section{picture}
\includegraphics[width=9cm,height=3cm]{135.png}
\end{center}

\hspace*{5cm} $Y_1 = X'y_2$ \\
\hspace*{5cm} $Y_2 = X + y'_1y_2$ \\
\hspace*{5cm} $z = Xy_1y'$ \\

\hspace*{0.5cm} $Y_1$ and $Y_2$ are the next states, which are the memory elements. These will be feedbacked to the input
as states $y_1$ and $y_2$ with some delay by D flip flop.\\
\hspace*{0.5cm} The circuit diagram for this sequence detector is given in Fig. 4.3.\\

\begin{center}
\section{picture}
\includegraphics[width=9cm,height=4cm]{135-2.png}
\end{center}

\newpage
\begin{flushleft}
    \textbf{136}\hspace*{0.1cm} \textbf{$|$} \hspace*{0.1cm} \texttt{Introduction to Automata Theory, Formal Languages and Computation}
  \end{flushleft}

  \vspace*{0.5cm}
\fcolorbox{red}{blue}{\textbf{\textcolor[rgb]{1.00,1.00,1.00}{Example 4.2}}}\hspace*{0.1cm} \texttt{Design a two-input two-output sequence detector which generates an output '1' every
time the sequence 1010 is detected. And for all other cases, output ‘0’ is generated.
Overlapping sequences are also counted.}\\

\vspace*{0.5cm}
\textbf{Solution:} The input string 1010100 is placed according to the clock pulses, and it looks like the
following.\\

  \begin{center}
\section{picture}
\includegraphics[width=8cm,height=1.7cm]{136-1.png}
\end{center}

And the output becomes as given earlier. Overlapping portion is 10, as shown in Fig. 4.4.\\
The state diagram is given in Fig. 4.5.\\

  \begin{center}
\section{picture}
\includegraphics[width=8cm,height=4cm]{136-2.png}
\end{center}

\textbf{State Table:} From the previous state diagram, a state table as follows can easily be constructed.\\

\begin{center}
\begin{tabular}{ccc}
 \hline

 \hline

 \hline

 \hline
 & \multicolumn{2}{c}{$Next State, O/P$}\\
 \cline{2-3}
 $State$ &  $X=0$ & $=1$\\
\hline
$S_1$ & $S_1, 0$  & $S_2, 0$\\
$S_2$ & $S_3, 0$  & $S_2, 0$\\
$S_3$ & $S_1, 0$  & $S_4, 0$\\
$S_4$ & $S_3, 1$  & $S_2, 0$\\
 \hline

 \hline

 \hline

 \hline
\end{tabular}
\end{center}

\textbf{State Assignments:} For making a digital circuit, the states must be assigned to some binary numbers to
make a digital circuit. This is called state assignment. As the number of states is 4, two-digit is sufficient
to represent the four states $(2^2 = 4)$.\\
\hspace*{0.5cm} Let $S_1$ be assigned to 00, $S_2$ be assigned to 01, $S_3$ be assigned to 11, $S_4$ be assigned to 10.

\end{document} 